\section{Movement Variability}


%\subsection{Why is important to study Human Movement Variability}
%%%%%%%%%%%%%%%%%%%%%%%%%%%%%%%%%%%%%%%%%%%%%%%%%%%%%%%%%
%{
%\paper{Xochicale 2019 in {\bf PhD thesis}}
%
%\begin{frame}{Why is challenging to investigate Human Movement Variability?}
%
%\Large
%Human movement variability not only involves multiple joints and 
%limbs for a specific task in a determined environment but also \\ 
%external information processed through: \\
%* all of our available senses and \\ 
%* our prior experiences.
%% (Xochicale, 2019).
%
%\end{frame}
%}
%

\subsection{}
%%%%%%%%%%%%%%%%%%%%%%%%%%%%%%%%%%%%%%%%%%%%%%%%%%%%%%%%
{
\paper{Bernstein 1967 in \textbf{The co-ordination and regulation of movements};
Newell and Vaillancourt 2001 in \textbf{Hum Mov Sci};
Davids et al. 2003 in \textbf{Sport Medicine}  
%Xochicale 2018 in {\bf Preprint PhD thesis (Zenodo)}
}
\begin{frame}{Modelling Movement Variability}
    \begin{figure}
        \includegraphicscopyright[width=1.0\linewidth]{davids2002/drawing}{}
	\caption{Newell's model of movement constrains} 
   \end{figure}
\end{frame}
}




\subsection{}
%%%%%%%%%%%%%%%%%%%%%%%%%%%%%%%%%%%%%%%%%%%%%%%%%%%%%%%%
{
\paper{Stergiou et al. 2006 in {\bf Neurologic Physical Therapy} 
Stergiou and Decker 2011 in {\bf Human Movement Science}}
\begin{frame}{Modelling Movement Variability}
    \begin{figure}
        \includegraphicscopyright[width=0.6\linewidth]{stergiou2006/pdf/drawing}{}
	\caption{Theoretical Model of Optimal Movement Variability} 
   \end{figure}
\end{frame}
}





\subsection{}
%%%%%%%%%%%%%%%%%%%%%%%%%%%%%%%%%%%%%%%%%%%%%%%%%%%%%%%%
{

\paper{Caballero et al. 2014 in {\bf European Journal of Human Movement}; 
Wijnants et al. 2009 in {\bf Nonlinear Dynamics, Psychology, and Life Sciences}
}

\begin{frame}{Nonlinear Analysis}

%Entropy measures to quantify regularity and complexity of time series.
%"Is there a best tool to measure variability?" (Caballero et al. 2014, p. 67)

%Some Tools from Nonlinear Analysis:

There is no best tool to quantify MV and unification of tools is still
an open question (Caballero et al. 2014; Wijnants et al. 2009)
which led me (i) to explore different nonlinear analysis 
to measure MV and (ii) to understand their strengths and weaknesses. 



\begin{itemize}
	\item Reconstructed State Space (Takens 1981)
	\item Recurrence Plots (Eckmann et al. 1987, Marwan et al. 2007)
	\item Approximate Entropy (Pincus 1991, 1995)
	\item Sample Entropy (Richman and Moorman, 2000)
	%\item Multiscale Entropy (Costa et al., 2002)
	%\item Detrended Fluctation Analysis (Peng et al., 1995)
	\item Largest Lyapunov exponent (Stergiou, 2016)
	\item Recurrence Quantification Analysis (Zbilut and Webber et al., 1992)
\end{itemize}


\end{frame}
}





%\section{Quantifying MV}

%\subsection{}
%%%%%%%%%%%%%%%%%%%%%%%%%%%%%%%%%%%%%%%%%%%%%%%%%%%%%%%%%
%{
%\begin{frame}{Modelling Movement Variability}
%
%\begin{itemize}
%	\item Entropy (Hatze, 1986)
%	\item Taks tolerance, noise reduction and covariation 
%		(Muller and Sternad, 2004) 
%	\item Hourglass model (Seifert et al., 2011)
%	\item variability for behaviour and nonlinear (Pretoni et al. 2010, 2013)
%\end{itemize}
%
%
%\end{frame}
%}
%

%\subsection{}
%%%%%%%%%%%%%%%%%%%%%%%%%%%%%%%%%%%%%%%%%%%%%%%%%%%%%%%%%
%{
%\begin{frame}{Nonlinear Analyses}
%
%\begin{itemize}
%	\item Approximate Entropy (Pincus 1991, 1995)
%	\item Sample Entropy (Richman and Moorman, 2000)
%	\item Multiscale Entropy (Costa et al., 2002)
%	\item Detrended Fluctation Analysis (Peng et al., 1995)
%	\item Recurrence Quantification Analysis (Zbilut and Webber et al., 1992)
%\end{itemize}
%
%
%\end{frame}
%}







%
%\subsection{Quantifying Movement Variability}
%%%%%%%%%%%%%%%%%%%%%%%%%%%%%%%%%%%%%%%%%%%%%%%%%%%%%%%%%
%{
%
%\paper{Stergiou 2006, Stergiou and Decker 2011 in {\bf Human Movement Science}}
%
%\begin{frame}{What is Movement Variably?}
%
%%\LARGE
%* MOVEMENT VARIABILITY (MV) is defined as the variations that occur in motor
%performance across multiple repetitions of a task.
%
%%\LARGE
%* Movement Variability is considered as an inherent feature 
%within and between each person's movement.   
%
%\end{frame}
%}
%
%
%

%\subsection{}
%%%%%%%%%%%%%%%%%%%%%%%%%%%%%%%%%%%%%%%%%%%%%%%%%%%%%%%%%
%{
%%\paper{}
%\begin{frame}{Modelling Movement Variability}
%    \begin{figure}
%        \includegraphicscopyright[width=0.8\linewidth]{stergiou2006/fig2A}{}
%%{Work in progress (Xochicale M. et al. 2018)}
%	\caption{MV within and between each person's movement.} 
%   \end{figure}
%\end{frame}
%}
%



%\subsection{}
%%%%%%%%%%%%%%%%%%%%%%%%%%%%%%%%%%%%%%%%%%%%%%%%%%%%%%%%%
%{
%%\paper{
%%Stergiou et al. 2006 in {\bf Neurologic Physical Therapy} 
%%Stergiou and Decker 2011 in {\bf Human Movement Science}}
%\begin{frame}{Modelling Movement Variability}
%    \begin{figure}
%        \includegraphicscopyright[width=0.9\linewidth]{hatze1986/drawing}{}
%	%\caption{Theoretical Model of Optimal Movement Variability} 
%   \end{figure}
%\end{frame}
%}
%
%
%\subsection{}
%%%%%%%%%%%%%%%%%%%%%%%%%%%%%%%%%%%%%%%%%%%%%%%%%%%%%%%%%
%{
%%\paper{
%%Stergiou et al. 2006 in {\bf Neurologic Physical Therapy} 
%%Stergiou and Decker 2011 in {\bf Human Movement Science}}
%\begin{frame}{Modelling Movement Variability}
%    \begin{figure}
%        \includegraphicscopyright[width=0.9\linewidth]{muller-sternad2004/drawing}{}
%	%\caption{Theoretical Model of Optimal Movement Variability} 
%   \end{figure}
%\end{frame}
%}
%
%
%\subsection{}
%%%%%%%%%%%%%%%%%%%%%%%%%%%%%%%%%%%%%%%%%%%%%%%%%%%%%%%%%
%{
%%\paper{
%%Stergiou et al. 2006 in {\bf Neurologic Physical Therapy} 
%%Stergiou and Decker 2011 in {\bf Human Movement Science}}
%\begin{frame}{Modelling Movement Variability}
%    \begin{figure}
%        \includegraphicscopyright[width=0.9\linewidth]{seifert2011/drawing}{}
%	%\caption{Theoretical Model of Optimal Movement Variability} 
%   \end{figure}
%\end{frame}
%}
%
%
%\subsection{}
%%%%%%%%%%%%%%%%%%%%%%%%%%%%%%%%%%%%%%%%%%%%%%%%%%%%%%%%%
%{
%%\paper{
%%Stergiou et al. 2006 in {\bf Neurologic Physical Therapy} 
%%Stergiou and Decker 2011 in {\bf Human Movement Science}}
%\begin{frame}{Modelling Movement Variability}
%    \begin{figure}
%        \includegraphicscopyright[width=0.9\linewidth]{preatoni2010/drawing}{}
%	%\caption{Theoretical Model of Optimal Movement Variability} 
%   \end{figure}
%\end{frame}
%}
%
%




%\subsection{}
%%%%%%%%%%%%%%%%%%%%%%%%%%%%%%%%%%%%%%%%%%%%%%%%%%%%%%%%%
%{
%%\paper{
%%Stergiou et al. 2006 in {\bf Neurologic Physical Therapy} 
%%Stergiou and Decker 2011 in {\bf Human Movement Science}}
%\begin{frame}{MV in Human-Humanoid Interaction}
%    \begin{figure}
%        \includegraphicscopyright[width=0.9\linewidth]{gorer2013/drawing}{}
%	%\caption{Theoretical Model of Optimal Movement Variability} 
%   \end{figure}
%\end{frame}
%}
%


%\subsection{}
%%%%%%%%%%%%%%%%%%%%%%%%%%%%%%%%%%%%%%%%%%%%%%%%%%%%%%%%%
%{
%%\paper{
%%Stergiou et al. 2006 in {\bf Neurologic Physical Therapy} 
%%Stergiou and Decker 2011 in {\bf Human Movement Science}}
%\begin{frame}{MV in Human-Humanoid Interaction}
%    \begin{figure}
%        \includegraphicscopyright[width=0.9\linewidth]{guneysu2014/drawing}{}
%   \end{figure}
%\end{frame}
%}
%

%\subsection{}
%%%%%%%%%%%%%%%%%%%%%%%%%%%%%%%%%%%%%%%%%%%%%%%%%%%%%%%%%
%{
%%\paper{
%%Stergiou et al. 2006 in {\bf Neurologic Physical Therapy} 
%%Stergiou and Decker 2011 in {\bf Human Movement Science}}
%\begin{frame}{Movement Variability in Human-Humanoid Interaction}
%    \begin{figure}
%        \includegraphicscopyright[width=0.9\linewidth]{guneysu2015/drawing}{}
%   \end{figure}
%\end{frame}
%}



%\subsection{}
%%%%%%%%%%%%%%%%%%%%%%%%%%%%%%%%%%%%%%%%%%%%%%%%%%%%%%%%%
%{
%%\paper{
%%Stergiou et al. 2006 in {\bf Neurologic Physical Therapy} 
%%Stergiou and Decker 2011 in {\bf Human Movement Science}}
%\begin{frame}{MV in Human-Robot Interaction}
%    \begin{figure}
%        \includegraphicscopyright[width=0.9\linewidth]{tsuchida2013/drawing}{}
%   \end{figure}
%\end{frame}
%}
%


%
%\subsection{}
%%%%%%%%%%%%%%%%%%%%%%%%%%%%%%%%%%%%%%%%%%%%%%%%%%%%%%%%%
%{
%%\paper{
%%Stergiou et al. 2006 in {\bf Neurologic Physical Therapy} 
%%Stergiou and Decker 2011 in {\bf Human Movement Science}}
%\begin{frame}{MV in Human-Robot Interaction}
%    \begin{figure}
%        \includegraphicscopyright[width=0.9\linewidth]{peng2015/drawing}{}
%   \end{figure}
%\end{frame}
%}
%


%
%\subsection{}
%%%%%%%%%%%%%%%%%%%%%%%%%%%%%%%%%%%%%%%%%%%%%%%%%%%%%%%%%
%{
%\paper{Vaillancourt and Newell 2002 in {\bf Neurobiology of Aging}}
%\begin{frame}{Modelling Movement Variability}
%    \begin{figure}
%        \includegraphicscopyright[width=0.7\linewidth]{vaillancourt2002/fig2A}{}
%	%{(Vaillancourt DE and Newell KM 2002 in {\bf Neurobiology of Aging})}
%	\caption{Theoretical Model of Optimal Movement Variability} 
%   \end{figure}
%\end{frame}
%}
%








\subsection{}
%%%%%%%%%%%%%%%%%%%%%%%%%%%%%%%%%%%%%%%%%%%%%%%%%%%%%%%%
{


\paper{Xochicale 2019 in {\bf PhD thesis}}

\begin{frame}{Research Questions}

\large
\begin{itemize}

\item What are the effects on RSSs, RPs, and RQA metrics
	of different embedding parameters, different recurrence thresholds 
	and different characteristics of time series 
	(structure, smoothness and window length size)?

\item What are the weaknesses and strengths of 
	RQA metrics when quantifying movement variability?

\item How does the smoothing of raw time series affect 
	methods of nonlinear analysis
	when quantifying movement variability?


\end{itemize}
%\badge{/badge/badge_v00}
\end{frame}
}


