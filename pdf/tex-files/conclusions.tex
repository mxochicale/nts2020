\section{Conclusions}

%\subsection{}
%%%%%%%%%%%%%%%%%%%%%%%%%%%%%%%%%%%%%%%%%%%%%%%%%%%%%%%%%
%{
%\begin{frame}{Conclusions}
%
%Modest contributions to knowledge
%\begin{itemize}
%	\item Measurements of Entropy using RQA appear
%	to be robust to real-word data (i.e. different time series
%	structures, window length size and levels of smoothness )
%	\item 3D surfaces of RQA are independent of either
%	the type series or the selection of parameters.
%	\item First open access thesis with data and code 
%	for its replication. 
%\end{itemize}
%
%%\badge{/badge/badge_v00}
%\end{frame}
%}
%


%
%\subsection{}
%%%%%%%%%%%%%%%%%%%%%%%%%%%%%%%%%%%%%%%%%%%%%%%%%%%%%%%%%
%{
%\begin{frame}{Concluding Remarks}
%
%\begin{itemize}
%	\item \textbf{ What to quantify in movement variability?} \\
%	\textit{Complexity of movement based on the degrees of freedom 
%	of a person performing a certain task in a defined environment.}
%
%	\item \textbf{ Which methods of nonlinear analysis are 
%	appropriate to quantify movement and how methods of nonlinear 
%	analysis are affected by real-world time series data?} \\
%	\textit{Shannon entropy using 3D surface plots of RQA 
%	appear to be robust to real-word data (i.e. different time series
%	structures, window length size and levels of smoothness).}
%%	\url{https://arxiv.org/abs/1810.09249}}
%
%	\item \textbf{ What are these techniques good for?} \\
%	\textit{Quantification of skill learning in HRI, 
%	dynamics of facial expressions, 
%	fetal behavioral development, or
%	nonlinear biomedical signal processing.}
%\end{itemize}
%
%%\badge{/badge/badge_v00}
%\end{frame}
%}
%
%


\subsection{}
%%%%%%%%%%%%%%%%%%%%%%%%%%%%%%%%%%%%%%%%%%%%%%%%%%%%%%%%
{
\paper{Xochicale 2019 in {\bf PhD thesis}}

\begin{frame}{Research Questions}

\begin{itemize}
	\item \textbf{ 
	Q1: What are the effects on RSSs, RPs, and RQA metrics
	of different embedding parameters, different recurrence thresholds 
	and different characteristics of time series 
	(structure, smoothness and window length size)?
	} \\
	\textit{
Nonlinear analysis tools can quantify different
data time-series. That said, the main contribution of this work 
is to find that measurements of entropy with 3D plot surfaces of RQA 
appear to be robust for real-word data (i.e. different time series
structures, window length size and levels of smoothness).
}

\end{itemize}

%\badge{/badge/badge_v00}
\end{frame}
}




\subsection{}
%%%%%%%%%%%%%%%%%%%%%%%%%%%%%%%%%%%%%%%%%%%%%%%%%%%%%%%%
{

\paper{Xochicale 2019 in {\bf PhD thesis}}
\begin{frame}{Research Questions}

\begin{itemize}
	\item \textbf{ 
	Q2: What are the weaknesses and strengths of 
	RQA metrics when quantifying movement variability?	
} \\
	\textit{
WEAKNESSES: (i) requirement of an expert for interpretation and 
computation of nonlinear analysis results, 
(ii) laborious implementation, and 
(iii) nonlinear analysis does not give the best representation
of the dynamics of time-series data. \\
STRENGTHS: (i) little setup of parameters for 3D plot surfaces, and
(ii) 3D plot surfaces can provide insight into the understanding
of the dynamics of time-series data.
}

\end{itemize}

%\badge{/badge/badge_v00}
\end{frame}
}



\subsection{}
%%%%%%%%%%%%%%%%%%%%%%%%%%%%%%%%%%%%%%%%%%%%%%%%%%%%%%%%
{

\paper{Xochicale 2019 in {\bf PhD thesis}}
\begin{frame}{Research Questions}

\begin{itemize}
	\item \textbf{ 
 	Q3: How does the smoothing of raw time series affect 
	methods of nonlinear analysis
	when quantifying movement variability?
} \\
	\textit{
Smoothing raw time series can create well defined trajectories
or patterns in RSS or RP, however such increase of smoothness 
can also create more complex (i.e. not well defined) trajectories
or patterns in nonlinear analysis. 
}

\end{itemize}

%\badge{/badge/badge_v00}
\end{frame}
}










\subsection{}
%%%%%%%%%%%%%%%%%%%%%%%%%%%%%%%%%%%%%%%%%%%%%%%%%%%%%%%%
{
\paper{Zia et al., 2017 in {\bf Computer Assisted Radiology and Surgery};
Mori 2012 in {\bf Development and Learning and Epigenetic Robotics};
Mitsukura et al., 2017 in {\bf Electroencephalography}; 
Marwan et al. 2019 in {\bf http://recurrence-plot.tk/}
}

\begin{frame}{Applications of Nonlinear Dynamics}
    \begin{figure}
        %\centering
        \includegraphicscopyright[width=0.99\linewidth]
	{applications/drawing}{}
	
	%\caption{(A) Normalised, (B) \textt{sgolay(p=5,n=25)}, and (C) \textt{sgolay(p=5,n=159)} } 
   \end{figure}
	
\end{frame}
}






\subsection{}
%%%%%%%%%%%%%%%%%%%%%%%%%%%%%%%%%%%%%%%%%%%%%%%%%%%%%%%%
{
\paper{Xochicale 2019 in {\bf PhD thesis}}

\begin{frame}{Future Work}

\textbf{Investigate}
\begin{itemize}
	\item other derivatives of acceleration data
	to have better understanding of the nature of human movement,
	\item other methodologies for state space reconstruction,
	\item the robustness of Entropy measurements with RQA, and 
	\item variability in perception of velocity.
\end{itemize}

\textbf{In the context of human-humanoid interaction,
the proposed method can be applied to} 
\begin{itemize}
	\item evaluate improvement of movement performance,
	\item provide feedback of level of skillfulness, and 
	\item quantify motor control problems and pathologies.
\end{itemize}


\end{frame}
}




\subsection{}
%%%%%%%%%%%%%%%%%%%%%%%%%%%%%%%%%%%%%%%%%%%%%%%%%%%%%%%%
{
%\paper{}

\begin{frame}{OA Publications}

\tiny
 
\textbf{PEER-REVIEW CONFERENCE PAPERS}
\begin{itemize}	
	\item \textit{Towards the Analysis of Movement Variability in Human-Humanoid Imitation Activities} 
	(HAI2017) 
	\item \textit{Towards the Quantification of Human-Robot Imitation Using Wearable Inertial Sensors} (HRI2017)
	\item \textit{Analysis of the Movement Variability in Dance Activities using Wearable Sensors} (WeRob2016)
	\item \textit{Understanding Movement Variability of Simplistic Gestures Using an Inertial Sensor} (PerDis2016)
\end{itemize}

\textbf{PREPRINTS \& in preparation}
\begin{itemize}	
	\item \textit{Strengths and weaknesses of Recurrence Quantification Analysis in the context of human-humanoid interaction}
	(ArXiv, October 2018) for Scientific Reports.
	\item \textit{3D surface plots of RQA Shannon Entropy} \\
 	for Frontiers in Applied Mathematics and Statistics.
\end{itemize}

\textbf{TALKS}
\begin{itemize}	
	\item \textit{Quantifying the Inherent Chaos of Human Movement Variability} \\
	15th Experimental Chaos and Complexity Conference 
	\item \textit{Towards the Analysis of Movement Variability for Facial Expressions with
	Nonlinear Dynamics} \\
	The 7th Consortium of European Research on Emotion Conference 
\end{itemize}

	
\end{frame}
}




\subsection{}
%%%%%%%%%%%%%%%%%%%%%%%%%%%%%%%%%%%%%%%%%%%%%%%%%%%%%%%%
{
\paper{Xochicale 2019 in {\bf PhD thesis}}

\begin{frame}{FIRST Open Access PhD Thesis at UoB (since 1900)}
    \begin{figure}
        %\centering
        \includegraphicscopyright[width=0.99\linewidth]
	{oathesis/drawing}{}	
	%\caption{ } 
   \end{figure}
	
\end{frame}
}



