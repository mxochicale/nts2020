\section{Results}

\subsection{}
%%%%%%%%%%%%%%%%%%%%%%%%%%%%%%%%%%%%%%%%%%%%%%%%%%%%%%%%%%
{
\paper{Xochicale 2019 in {\bf PhD Thesis}}
\begin{frame}{From Raw to Smoothed Time Series}
   \begin{figure}
       %\centering
        \includegraphicscopyright[width=0.9\linewidth]
	{results/ch6-ts/fig_6_01}{}
	\caption{Time-series of horizontal movements for 
	(A) normalised, (B) \textt{sgolay(p=5,n=25)}, and 
	(C) \textt{sgolay(p=5,n=159).} } 
   \end{figure}
\end{frame}
}


\subsection{}
%%%%%%%%%%%%%%%%%%%%%%%%%%%%%%%%%%%%%%%%%%%%%%%%%%%%%%%%%%
{
\paper{Xochicale 2019 in {\bf PhD Thesis}}
\begin{frame}{From Raw to Smoothed Time Series}
   \begin{figure}
       %\centering
        \includegraphicscopyright[width=0.9\linewidth]
	{results/ch6-ts/fig_6_02}{}
	\caption{Time-series of vertical movements for 
	(A) normalised, (B) \textt{sgolay(p=5,n=25)}, and 
	(C) \textt{sgolay(p=5,n=159).} } 
   \end{figure}
\end{frame}
}



\subsection{}
%%%%%%%%%%%%%%%%%%%%%%%%%%%%%%%%%%%%%%%%%%%%%%%%%%%%%%%%%
{
\paper{Xochicale 2019 in {\bf PhD Thesis}}
\begin{frame}{Minimum Embedding Parameters}
    \begin{figure}
        %\centering
        \includegraphicscopyright[width=0.9\linewidth]
	{results/ch6-empa/fig_6_03}{}
	\caption{(A) Minimum Embedding Dimension 
		 (B) First Minimum AMI
		}  
   \end{figure}
	
\end{frame}
}


\subsection{}
%%%%%%%%%%%%%%%%%%%%%%%%%%%%%%%%%%%%%%%%%%%%%%%%%%%%%%%%
{
%\paper{Xochicale 2019 in {\bf PhD Thesis}}
\begin{frame}{Reconstructed State Spaces}
    \begin{figure}
        %\centering
        \includegraphicscopyright[width=0.9\linewidth]
	{results/ch6-rss/fig_6_04_6_05}{}
	\caption{RSS for participant 01 computed with ($m=6$, $\tau=8$)
	for different activities, signals and source of time-series data.
	} 
   \end{figure}
\end{frame}
}





\subsection{}
%%%%%%%%%%%%%%%%%%%%%%%%%%%%%%%%%%%%%%%%%%%%%%%%%%%%%%%
{
%\paper{Xochicale 2019 in {\bf PhD Thesis}}
\begin{frame}{Recurrence Plots}
    \begin{figure}
        %\centering
        \includegraphicscopyright[width=0.9\linewidth]
	{results/ch6-rp/fig_6_06_6_07}{}
	\caption{RP for participant 01 computed 
	with ($m=6$, $\tau=8$, $\epsilon=1$)
	for different activities, signals and source of time-series data.
	} 
   \end{figure}
	
\end{frame}
}




\subsection{}
%%%%%%%%%%%%%%%%%%%%%%%%%%%%%%%%%%%%%%%%%%%%%%%%%%%%%%%
{
\paper{Xochicale 2019 in {\bf PhD Thesis}}

\begin{frame}{Recurrence Quantification Analysis}
    \begin{figure}
        %\centering
        \includegraphicscopyright[width=0.5\linewidth]
	{results/ch6-rqa/fig_6_08}{}
	\caption{Box values of  RQA computed with 
	($m=7$, $\tau=5$, $\epsilon=1$). 
	These values are for 20 participants.
} 
   \end{figure}
	
\end{frame}
}


\subsection{}
%%%%%%%%%%%%%%%%%%%%%%%%%%%%%%%%%%%%%%%%%%%%%%%%%%%%%%%
{
\paper{Xochicale 2019 in {\bf PhD Thesis}}

\begin{frame}{RQA ENTR for $\epsilon$ thresholds
	\& smoothness
	{\bf (in preparation)}
}
    \begin{figure}
        %\centering
        \includegraphicscopyright[width=0.5\linewidth]
	{results/inprogress/rqa_epsilons}{}
	\caption{
	RQA ENTR values are for
	$p03$, sensor $HS01$, of a window size of 10-secs (500 samples).
} 
   \end{figure}
	
\end{frame}
}



\subsection{}
%%%%%%%%%%%%%%%%%%%%%%%%%%%%%%%%%%%%%%%%%%%%%%%%%%%%%%%
{
\paper{Xochicale 2019 in {\bf PhD Thesis}}

\begin{frame}{RQA ENTR for sensors and activities.
	{\bf (in preparation)}
}
    \begin{figure}
        %\centering
        \includegraphicscopyright[width=0.9\linewidth]
	{results/inprogress/rqa_sensors_activities}{}
	\caption{
	RQA ENTR values are for
	$p03$, $sg0$ and window size of 10-secs (500 samples).
} 
   \end{figure}
	
\end{frame}
}










\subsection{}
%%%%%%%%%%%%%%%%%%%%%%%%%%%%%%%%%%%%%%%%%%%%%%%%%%%%%%%%
{
\paper{Xochicale 2019 in {\bf PhD Thesis}}

\begin{frame}{3D surfaces plots of RQA
	{\bf (in preparation)}
}
    \begin{figure}
        %\centering
        \includegraphicscopyright[width=0.99\linewidth]
	{results/ch6-3drqa/fig_6_09}{}
	\small
	\caption{
		3D RQA surfaces 
	with increasing pair of embedding parameters 
	($0 \le m \le 10$, $0 \le \tau \le 10$) 
	and recurrence thresholds ($ 0.2 \le \epsilon \le 3 $).
	} 
   \end{figure}
	
\end{frame}
}




\subsection{}
%%%%%%%%%%%%%%%%%%%%%%%%%%%%%%%%%%%%%%%%%%%%%%%%%%%%%%%%
{
\paper{Xochicale 2019 in {\bf PhD Thesis}}

\begin{frame}{Sensors and activities
	{\bf (in preparation)}
}
    \begin{figure}
        %\centering
         \includegraphicscopyright[width=0.99\linewidth]
	{results/ch6-3drqa-sensors/fig_6_10_6_11}{}
	\caption{3D surface plots of RQA for 
	different sensors and activities.} 
   \end{figure}
	
\end{frame}
}






\subsection{}
%%%%%%%%%%%%%%%%%%%%%%%%%%%%%%%%%%%%%%%%%%%%%%%%%%%%%%%
{
\paper{Xochicale 2019 in {\bf PhD Thesis}}

\begin{frame}{Window size lengths
	{\bf (in preparation)}
}
   \begin{figure}
        %\centering
        \includegraphicscopyright[width=0.55\linewidth]
	{results/ch6-3drqa-windows/fig_6_12}{}
	\caption{Window length size effect on 3D surface plots of RQA.} 
   \end{figure}
	
\end{frame}
}





\subsection{}
%%%%%%%%%%%%%%%%%%%%%%%%%%%%%%%%%%%%%%%%%%%%%%%%%%%%%%%
{
\paper{Xochicale 2019 in {\bf PhD Thesis}}

\begin{frame}{Smoothness
	{\bf (in preparation)}
}
    \begin{figure}
        %\centering
         \includegraphicscopyright[width=0.45\linewidth]
	{results/ch6-3drqa-smoothness/fig_6_13}{}
	\caption{Smoothness effect on 3D surface plots of RQA.} 
   \end{figure}
	
\end{frame}
}




\subsection{}
%%%%%%%%%%%%%%%%%%%%%%%%%%%%%%%%%%%%%%%%%%%%%%%%%%%%%%%
{
\paper{Xochicale 2019 in {\bf PhD Thesis}}

\begin{frame}{Participants
	{\bf (in preparation)}
}
    \begin{figure}
        %\centering
         \includegraphicscopyright[width=0.55\linewidth]
	{results/ch6-3drqa-participants/fig_6_14}{}
	\caption{Participants differences of 3D surface plots of RQA.} 
   \end{figure}
	
\end{frame}
}

%
%
%\subsection{}
%%%%%%%%%%%%%%%%%%%%%%%%%%%%%%%%%%%%%%%%%%%%%%%%%%%%%%%%%
%{
%%\paper{}
%
%\begin{frame}{Minimum Embedding Parameters}
%    \begin{figure}
%        %\centering
%        \includegraphicscopyright[width=0.5\linewidth]
%	{results/summary/drawing}{}
%	\caption{(A) Minimum Embedding Dimension 
%		 (B) First Minimum AMI
%		}  
%   \end{figure}
%	
%\end{frame}
%}
%

